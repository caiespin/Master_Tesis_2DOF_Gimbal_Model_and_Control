%%%%%%%%%%%%%%%%%%%%%%%%%%%%%%%%%%%%%%%%%%%%%%%%%%%%%%%%%%%%%%%%%%%%%%%%%%%%%%%%%%%%%%%%%%%%%%
% Plantilla para tesis por David Luna se encuentra bajo una Licencia
% Creative Commons Atribuci�n-NoComercial-CompartirIgual 3.0 Unported.
% http://creativecommons.org/licenses/by-nc-sa/3.0/
%%%%%%%%%%%%%%%%%%%%%%%%%%%%%%%%%%%%%%%%%%%%%%%%%%%%%%%%%%%%%%%%%%%%%%%%%%%%%%%%%%%%%%%%%%%%%%

%\addcontentsline{toc}{chapter}{Resumen}
\begin{center}
\textbf{\large RESUMEN}
\end{center}

\vspace{3cm}

Esta tesis trata sobre el modelado, control y desarrollo de una unidad de c�mara Pan-Tilt (gimbal) de 2 grados de libertad el cual esta destinado a usarse en un avi�n no tripulado (UAV) para fines de inteligencia, vigilancia y reconocimiento. El objetivo de esta tesis es desarrollar una estructura de control que permita estabilizar el eje �ptico de la c�mara hacia un objetivo aislando la c�mara de las perturbaciones del entorno de operaci�n, tales como torques externos y el movimiento angular de la plataforma logrando as� mantener su direcci�n en el espacio inercial.\\

Se desarrollan las ecuaciones de movimiento del gimbal por el m�todo de Newton-Euler y se formulan para poder interpretar las propiedades del sistema. Con ayuda de software CAD se obtiene una aproximaci�n de los par�metros inerciales y se realizan diferentes simulaciones. Se procede a la construcci�n del prototipo y la programaci�n del micro controlador utilizado. Finalmente se realizan pruebas para la obtenci�n y verificaci�n de resultados.\\

\newpage
$\ $
\thispagestyle{empty} % para que no se numere esta pagina
