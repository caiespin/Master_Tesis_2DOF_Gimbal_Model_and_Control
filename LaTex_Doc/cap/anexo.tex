%%%%%%%%%%%%%%%%%%%%%%%%%%%%%%%%%%%%%%%%%%%%%%%%%%%%%%%%%%%%%%%%%%%%%%%%%%%%%%%%%%%%%%%%%%%%%%
% Plantilla para tesis por David Luna se encuentra bajo una Licencia
% Creative Commons Atribuci�n-NoComercial-CompartirIgual 3.0 Unported.
% http://creativecommons.org/licenses/by-nc-sa/3.0/
%%%%%%%%%%%%%%%%%%%%%%%%%%%%%%%%%%%%%%%%%%%%%%%%%%%%%%%%%%%%%%%%%%%%%%%%%%%%%%%%%%%%%%%%%%%%%%

\chapter{C\'{a}lculos del Modelado}

\section{} \label{sec:A1}

Usando las ecuaciones desarrolladas en la secci\'{o}n ~\ref{sec:RelCinBas} podemos obtener relaciones de las velocidades angulares en t\'{e}rminos de las variables controladas y las perturbaciones de la base. De la ecuaci\'{o}n ~\ref{eq:w_Tz}
\begin{equation*}
\omega_{T_z} = s \varepsilon \, \omega_{P_x} + c \varepsilon \, \omega_{P_z}
\end{equation*}

Despejando $\omega_{P_z}$ de la expresi\'{o}n anterior, obtenemos
\begin{equation}
\omega_{P_z}  = \frac{1}{c \varepsilon} \left( \omega_{T_z} - s \varepsilon \, \omega_{P_x} \right)
\label{eq:w_Pz_cv}
\end{equation}

Sustituyendo ~\ref{eq:w_Pz_cv} en la ecuaci\'{o}n ~\ref{eq:w_Tx}
\begin{align*}
\omega_{T_x} &=c \varepsilon \, \omega_{P_x} - s \varepsilon \, \left( \frac{\omega_{T_z} - s \varepsilon \, \omega_{P_x}}{c \varepsilon} \right) \\
 &=c \varepsilon \, \omega_{P_x} -  \left( \frac{s \varepsilon \, \omega_{T_z} - s^2 \varepsilon \, \omega_{P_x}}{c \varepsilon} \right) \\
 &= \frac{c^2 \varepsilon \, \omega_{P_x} - s \varepsilon \, \omega_{T_z} - s^2 \varepsilon \, \omega_{P_x}}{c \varepsilon} \\
 &=  \frac{ \left( c^2 \varepsilon + s^2 \varepsilon \right) \omega_{P_x} - s \varepsilon \, \omega_{T_z}}{c \varepsilon}  
\end{align*}

\begin{equation}
\omega_{T_x}  = \frac{1}{c \varepsilon} \left( \omega_{P_x} - s \varepsilon \, \omega_{T_z} \right)
\label{eq:w_Tx_cv}
\end{equation}

\section{} \label{sec:A2}

La relaci�n entre las aceleraciones angulares del eslab\'{o}n interno y el externo del gimbal, pueden obtenerse derivando la ecuaci\'{o}n ~\ref{eq:w_T}

\begin{equation*}
\frac{d \, \omega _{T}}{dt} =\frac{d}{dt}\left(R_{T}^{P}\omega _{P}+\dot{\varepsilon }\hat{\jmath}_{T}\right)
\end{equation*}

Derivando la expresi\'{o}n anterior, obtenemos

\begin{equation}
\dot{\omega} _{T} =R_{T}^{P}\dot{\omega} _{P} + R_{Aux}\omega _{P}\dot{\varepsilon } + \ddot{E}
\label{eq:dotw_T}
\end{equation}

Donde

\begin{equation*}
R_{Aux}=  \left[ 
\begin{array}{ccc}
-s\varepsilon & 0 & -c\varepsilon \\ 
0 & 0 & 0 \\ 
c\varepsilon  & 0& -s\varepsilon%
\end{array}%
\right]
\end{equation*}

y

\begin{equation*}
\ddot{E}=\left[ \begin{array}{c}  0 \\ \ddot{\varepsilon} \\ 0 \end{array} \right]
\end{equation*}

Resolviendo~\ref{eq:dotw_T} para $\dot{\omega} _{P}$

\begin{equation*}
\begin{array}{rcl}
R_{T}^{P}\dot{\omega} _{P}&=&\dot{\omega} _{T} - R_{Aux}\omega _{P}\dot{\varepsilon } - \ddot{E}\\
 \dot{\omega} _{P}&=&{R_{T}^{P}}^{-1}\left[\dot{\omega} _{T} - R_{Aux}\omega _{P}\dot{\varepsilon } - \ddot{E} \right]
 \end{array}
\end{equation*}

Aplicando las propiedades de la matriz de rotaci�n ${R_{T}^{P}}^{-1}={R_{T}^{P}}^{T}=R_{P}^{T}$, obtenemos

\begin{equation}
\dot{\omega} _{P}=R_{P}^{T}\dot{\omega} _{T} - R_{P}^{T}R_{Aux}\omega _{P}\dot{\varepsilon } - R_{P}^{T}\ddot{E}
\label{eq:dotw_P}
\end{equation}

\section{} \label{sec:A3}

La din\'{a}mica de la \textit{Elevaci\'{o}n cruzada} esta dada por el tercer elemento del vector ~\ref{CrossEl_4}

\begin{equation*}
\begin{array}{rcl}
M_P&=&I_{P} R_{P}^{T}\dot{\omega}_{T} + R_{P}^{T} I_{T}\dot{\omega}_{T} + \omega_P \times I_{P}\omega_P  + R_{P}^{T}\left(\omega _{T} \times I_{T}\omega_T \right)\\
& & - I_{P}R_{P}^{T}R_{Aux}\omega _{P}\dot{\varepsilon } - I_{P} R_{P}^{T}\ddot{E} 
\end{array}
\end{equation*}

Ahora, expandiendo el primer t\'{e}rmino del lado derecho

\begin{equation*}
\begin{array}{rcl}
I_{P} R_{P}^{T}\dot{\omega}_{T}&=&\left[ 
\begin{array}{ccc}
I_{P_x}  &      0   &   0 \\
   0     &  I_{P_y} &   0 \\
   0     &      0   & I_{P_z}\\
\end{array} \right]\left[  \begin{array}{ccc} c\varepsilon & 0 & s\varepsilon \\  0 & 1 & 0 \\  -s\varepsilon \text{\ \ \ } & 0\text{\ \ } & c\varepsilon \end{array}\right]\left[\begin{array}{c} \dot{\omega}_{T_x} \\ \dot{\omega}_{T_y} \\ \dot{\omega}_{T_z} \end{array}\right]\\
& & + \left[  \begin{array}{ccc} c\varepsilon & 0 & s\varepsilon \\  0 & 1 & 0 \\  -s\varepsilon \text{\ \ \ } & 0\text{\ \ } & c\varepsilon \end{array}\right]\left[ 
\begin{array}{ccc}
I_{T_x}  &      0   &   0 \\
   0     &  I_{T_y} &   0 \\
   0     &      0   & I_{T_z}\\
\end{array} \right]\left[\begin{array}{c} \dot{\omega}_{T_x} \\ \dot{\omega}_{T_y} \\ \dot{\omega}_{T_z} \end{array}\right]
\end{array}
\end{equation*}

\begin{equation*}
\begin{array}{rcl}
 &=&\left[ 
\begin{array}{ccc}
I_{P_x}c\varepsilon   &      0   &   I_{P_x} s\varepsilon \\
   0     &  I_{P_y} &   0 \\
-I_{P_z} s\varepsilon     &      0   & I_{P_z} c\varepsilon \\
\end{array} \right]\left[\begin{array}{c} \dot{\omega}_{T_x} \\ \dot{\omega}_{T_y} \\ \dot{\omega}_{T_z} \end{array}\right]\\
& & + \left[ 
\begin{array}{ccc}
I_{T_x}c\varepsilon   &      0   &   I_{T_Z} s\varepsilon \\
   0     &  I_{T_y} &   0 \\
-I_{T_x} s\varepsilon     &      0   & I_{T_z} c\varepsilon \\
\end{array} \right]\left[\begin{array}{c} \dot{\omega}_{T_x} \\ \dot{\omega}_{T_y} \\ \dot{\omega}_{T_z} \end{array}\right]
\end{array}
\end{equation*}

Tomando el tercer t\'{e}rmino 

\begin{equation*}
\left[ I_{P} R_{P}^{T}\dot{\omega}_{T} \right]_3  =  -I_{P_z} s\varepsilon \dot{\omega}_{T_x} + I_{P_z} c\varepsilon \dot{\omega}_{T_z}\dot{\omega}_{T_z} - I_{T_x} s\varepsilon \dot{\omega}_{T_x} + I_{T_z} c\varepsilon \dot{\omega}_{T_z}
\end{equation*}

\begin{equation}
\left[ I_{P} R_{P}^{T}\dot{\omega}_{T} \right]_3 = c\varepsilon \left( I_{P_z}+I_{T_z} \right)\dot{\omega}_{T_z} - s\varepsilon \left( I_{P_z}+I_{T_x} \right)\dot{\omega}_{T_x}
\label{eq:Term_1}
\end{equation}

Expandiendo el segundo y tercer t\'{e}rmino

\begin{equation*}
\begin{array}{rcl}
\omega_P \times I_{P}\omega_P  + R_{P}^{T}\left(\omega _{T} \times I_{T}\omega_T \right)&=& \left[\begin{array}{c} \left( I_{P_z}-I_{P_y} \right) \omega_{P_y} \omega_{P_z} \\ 
\left( I_{P_x}-I_{P_z} \right) \omega_{P_x} \omega_{P_z} \\ \left( I_{P_y}-I_{P_x} \right) \omega_{P_x} \omega_{P_y} \end{array}\right]\\
& & +\left[ 
\begin{array}{ccc}
I_{T_x}c\varepsilon   &      0   &   I_{T_Z} s\varepsilon \\
   0     &  I_{T_y} &   0 \\
-I_{T_x} s\varepsilon     &      0   & I_{T_z} c\varepsilon \\
\end{array} \right] \left[\begin{array}{c} \left( I_{T_z}-I_{T_y} \right) \omega_{T_y} \omega_{T_z} \\ 
\left( I_{T_x}-I_{T_z} \right) \omega_{T_x} \omega_{T_z} \\ \left( I_{T_y}-I_{T_x} \right) \omega_{T_x} \omega_{T_y} \end{array}\right]\\
 & = &\left[\begin{array}{c} \left( I_{P_z}-I_{P_y} \right) \omega_{P_y} \omega_{P_z} \\ 
\left( I_{P_x}-I_{P_z} \right) \omega_{P_x} \omega_{P_z} \\ \left( I_{P_y}-I_{P_x} \right) \omega_{P_x} \omega_{P_y} \end{array}\right] \\
 & &+ \left[\begin{array}{c} c\varepsilon \left( I_{T_z}-I_{T_y} \right) \omega_{T_y} \omega_{T_z} + s\varepsilon \left( I_{T_y}-I_{T_x} \right) \omega_{T_x} \omega_{T_y}\\ 
\left( I_{T_x}-I_{T_z} \right) \omega_{T_x} \omega_{T_z} \\  -s\varepsilon \left( I_{T_z}-I_{T_y} \right) \omega_{T_y} \omega_{T_z} + c\varepsilon \left( I_{T_y}-I_{T_x} \right) \omega_{T_x} \omega_{T_y} \end{array}\right]
\end{array}
\end{equation*}

Tomando el tercer t\'{e}rmino 

\begin{equation*}
\begin{array}{cl}
\left[ \omega_P \times I_{P}\omega_P  + R_{P}^{T}\left(\omega _{T} \times I_{T}\omega_T \right) \right]_3 = &\left( I_{P_y}-I_{P_x} \right) \omega_{P_x} \omega_{P_y} -s\varepsilon \left( I_{T_z}-I_{T_y} \right) \omega_{T_y} \omega_{T_z}\\
 & + c\varepsilon \left( I_{T_y}-I_{T_x} \right) \omega_{T_x} \omega_{T_y}
\end{array}
\end{equation*}

Sustituyendo ~\ref{eq:w_Tx_cv} y desarrollando, obtenemos

\begin{equation}
\begin{array}{cl}
\left[ \omega_P \times I_{P}\omega_P  + R_{P}^{T}\left(\omega _{T} \times I_{T}\omega_T \right) \right]_3 = & \left( I_{P_y}-I_{P_x} \right) \omega_{P_x} \omega_{P_y} - s\varepsilon I_{T_z} \omega_{T_y} \omega_{T_z}\\
 & + I_{T_y} \omega_{T_y} \omega_{P_x} - c\varepsilon I_{T_x} \omega_{T_x} \omega_{T_y}\end{array}
 \label{eq:Term_2_3}
\end{equation}\newline

Expandiendo el cuarto t\'{e}rmino

\begin{equation*}
\begin{array}{rcl}
I_{P}R_{P}^{T}R_{Aux}\omega _{P}\dot{\varepsilon }&=& \left[ 
\begin{array}{ccc}
I_{T_x}  &      0   &   0 \\
   0     &  I_{T_y} &   0 \\
   0     &      0   & I_{T_z}\\
\end{array} \right]\left[  \begin{array}{ccc} c\varepsilon & 0 & s\varepsilon \\  0 & 1 & 0 \\  -s\varepsilon \text{\ \ \ } & 0\text{\ \ } & c\varepsilon \end{array}\right]\\

 & & \left[ 
\begin{array}{ccc}
-s\varepsilon & 0 & c\varepsilon \\ 
0 & 0 & 0 \\ 
c\varepsilon  & 0& -s\varepsilon%
\end{array}%
\right]\left[\begin{array}{c} \omega_{P_x} \\ \omega_{P_y} \\ \omega_{P_z} \end{array}\right]\dot{\varepsilon }\\
 
 &=& \left[ 
\begin{array}{ccc}
I_{P_x}c\varepsilon   &      0   &   I_{P_x} s\varepsilon \\
   0     &  I_{P_y} &   0 \\
-I_{P_z} s\varepsilon     &      0   & I_{P_z} c\varepsilon \\
\end{array} \right]\left[ 
\begin{array}{ccc}
-s\varepsilon & 0 & c\varepsilon \\ 
0 & 0 & 0 \\ 
c\varepsilon  & 0& -s\varepsilon%
\end{array}%
\right]\left[\begin{array}{c} \omega_{P_x} \\ \omega_{P_y} \\ \omega_{P_z} \end{array}\right]\dot{\varepsilon }\\

&=& \left[ 
\begin{array}{ccc}
-s\varepsilon c\varepsilon I_{P_x} + s\varepsilon c\varepsilon I_{P_x} & 0 & -c^2\varepsilon I_{P_x} - s^2\varepsilon I_{P_x} \\ 
0 & 0 & 0 \\ 
 s^2\varepsilon I_{P_z} + c^2\varepsilon I_{P_z} & 0& s\varepsilon c\varepsilon I_{P_z} - s\varepsilon c\varepsilon I_{P_z}%
\end{array}%
\right]\left[\begin{array}{c} \omega_{P_x} \\ \omega_{P_y} \\ \omega_{P_z} \end{array}\right]\dot{\varepsilon }\\

&=& \left[ 
\begin{array}{ccc}
0 & 0 & -\left( c^2\varepsilon + s^2\varepsilon \right) I_{P_x} \\ 
0 & 0 & 0 \\ 
 \left( c^2\varepsilon + s^2\varepsilon \right) I_{P_z} & 0& 0%
\end{array}%
\right]\left[\begin{array}{c} \omega_{P_x} \\ \omega_{P_y} \\ \omega_{P_z} \end{array}\right]\dot{\varepsilon }\\

&=& \left[ 
\begin{array}{ccc}
0 & 0 & -I_{P_x} \\ 
0 & 0 & 0 \\ 
I_{P_z} & 0& 0%
\end{array}%
\right]\left[\begin{array}{c} \omega_{P_x} \\ \omega_{P_y} \\ \omega_{P_z} \end{array}\right]\dot{\varepsilon }\\

&=& \left[\begin{array}{c} -I_{P_x} \omega_{P_x}\dot{\varepsilon } \\ 0 \\ I_{P_z} \omega_{P_z}\dot{\varepsilon } \end{array}\right]
\end{array}
\end{equation*}

Tomando el tercer t\'{e}rmino

\begin{equation}
\left[ I_{P}R_{P}^{T}R_{Aux}\omega _{P}\dot{\varepsilon}\right]_3 = I_{P_z} \omega_{P_z}\dot{\varepsilon}
\label{eq:Term_4}
\end{equation}\newline

Expandiendo el quinto t\'{e}rmino

\begin{equation*}
\begin{array}{rcl}
I_{P} R_{P}^{T}\ddot{E}&=&\left[ 
\begin{array}{ccc}
I_{P_x}  &      0   &   0 \\
   0     &  I_{P_y} &   0 \\
   0     &      0   & I_{P_z}\\
\end{array} \right]\left[  \begin{array}{ccc} c\varepsilon & 0 & s\varepsilon \\  0 & 1 & 0 \\  -s\varepsilon \text{\ \ \ } & 0\text{\ \ } & c\varepsilon \end{array}\right]\left[\begin{array}{c} 0 \\ \ddot{\varepsilon} \\ 0 \end{array}\right]\\
 &=&\left[ 
\begin{array}{ccc}
I_{P_x}c\varepsilon   &      0   &   I_{P_x} s\varepsilon \\
   0     &  I_{P_y} &   0 \\
-I_{P_z} s\varepsilon     &      0   & I_{P_z} c\varepsilon \\
\end{array} \right]\left[\begin{array}{c} 0 \\ \ddot{\varepsilon} \\ 0 \end{array}\right]\\

 &=& \left[\begin{array}{c} 0 \\ I_{P_y}\ddot{\varepsilon} \\ 0 \end{array}\right]
\end{array}
\end{equation*}

Del desarrollo anterior podemos observar que el tercer t\'{e}rmino es cero

\begin{equation}
\left[I_{P} R_{P}^{T}\ddot{E}\right]_3=0
\label{eq:Term_5}
\end{equation}

\section{} \label{sec:A4}

La ecuaci\'{o}n ~\ref{CrossEl_6}, dada en la secci\'{o}n ~\ref{sec:Din_cross}, puede ser modificada al expresar las variables $\omega_{T_x}$ y $\omega_{P_z}$ en t\'{e}rminos de las variables de control y las perturbaciones de la base, estas variables ya fueron desarrolladas en  ~\ref{sec:A1}.

Usando las ecuaciones ~\ref{eq:w_Tx_cv} y ~\ref{eq:w_Pz_cv}, el primer t\'{e}rmino del lado derecho de ~\ref{CrossEl_7}, puede ser expresado como

\begin{equation}
\begin{array}{rcl}
s\varepsilon \left( I_{P_z}-I_{T_x} \right)\dot{\omega}_{T_x} & = & s\varepsilon \left( I_{P_z}+I_{T_x} \right) \left[c\varepsilon \dot{\omega}_{B_x} + s\varepsilon \dot{\omega}_{B_y} - t\varepsilon \omega_{P_y} \omega_{P_x} - \omega_{P_y} \omega_{B_z} \right] \\
 & & + t\varepsilon \left( I_{P_z}+I_{T_x} \right) \omega_{P_y} \omega_{T_z}
\end{array}
\end{equation}

Desarrollando el segundo t\'{e}rmino, obtenemos

\begin{equation}
\begin{array}{rcl}
\left[ I_{P_z} \omega_{T_x} - s\varepsilon I_{T_x} \omega_{P_z} \right]\dot{\varepsilon}&=&t\varepsilon \left( I_{P_z}+I_{T_x} \right)\omega_{T_z} \omega_{P_y}\\
& & - t\varepsilon \left( I_{P_z}+I_{T_x} \right)\omega_{T_z} \omega_{T_y} + \frac{1}{c\varepsilon} \left( I_{P_z}+ s^2\varepsilon I_{T_x} \right)\omega_{T_y} \omega_{P_x}\\
 & & - \frac{1}{c\varepsilon} \left( I_{P_z}+ s^2\varepsilon I_{T_x} \right)\omega_{P_x} \omega_{P_y}
\end{array}
\end{equation}

Desarrollando el tercer t\'{e}rmino

\begin{equation}
\begin{array}{c}
-c\varepsilon\left[ \left( I_{P_y}-I_{P_x} \right) \omega_{P_x} \omega_{P_y}- s\varepsilon I_{T_z} \omega_{T_y} \omega_{T_z} + I_{T_y} \omega_{T_y} \omega_{P_x} - c\varepsilon I_{T_x} \omega_{T_x} \omega_{T_y}\right]=\\
c\varepsilon\left( I_{T_x}-I_{T_y} \right)\omega_{T_y} \omega_{P_x} + s\varepsilon \, c\varepsilon\left( I_{T_z}-I_{T_x} \right)\omega_{T_y} \omega_{T_z} - c\varepsilon\left( I_{P_y}-I_{P_x} \right) \omega_{P_x} \omega_{P_y}
\end{array}
\end{equation}

Sustituyendo todo lo anterior en ~\ref{CrossEl_7}, obtenemos

\begin{equation}
\begin{array}{rcl}
Js\dot{\omega}_{T_z}&=&s\varepsilon \left( I_{P_z}+I_{T_x} \right) \left[c\varepsilon \dot{\omega}_{B_x} + s\varepsilon \dot{\omega}_{B_y} - t\varepsilon \omega_{P_y} \omega_{P_x} - \omega_{P_y} \omega_{B_z} \right] \\
 & & + t\varepsilon \left( I_{P_z}+I_{T_x} \right) \omega_{P_y} \omega_{T_z}t\varepsilon \left( I_{P_z}+I_{T_x} \right)\omega_{T_z} \omega_{P_y}\\
& & - t\varepsilon \left( I_{P_z}+I_{T_x} \right)\omega_{T_z} \omega_{T_y} + \frac{1}{c\varepsilon} \left( I_{P_z}+ s^2\varepsilon I_{T_x} \right)\omega_{T_y} \omega_{P_x}\\
 & & - \frac{1}{c\varepsilon} \left( I_{P_z}+ s^2\varepsilon I_{T_x} \right)\omega_{P_x} \omega_{P_y} + c\varepsilon\left( I_{T_x}-I_{T_y} \right)\omega_{T_y} \omega_{P_x}\\

 & & + s\varepsilon \, c\varepsilon\left( I_{T_z}-I_{T_x} \right)\omega_{T_y} \omega_{T_z} - c\varepsilon\left( I_{P_y}-I_{P_x} \right) \omega_{P_x} \omega_{P_y}\\
 
 & & + c\varepsilon\left[T_{U_{Pz}}-k_{Pvf}\dot{\eta }+k_{Pcf} \, \mathrm{sign}\left( \dot{\eta} \right) + T_{az} \right]
 \end{array}
\end{equation}

Reagrupando y desarrollando la expresi\'{o}n anterior

\begin{equation}
\begin{array}{rcl}
Js\dot{\omega}_{T_z} & = & -k_{Pvf} \omega_{T_z} +\left[s\varepsilon \, c\varepsilon\left(I_{T_z} -I_{T_x} \right) - t\varepsilon \left(I_{P_z} +I_{T_x} \right) \right]\omega_{T_y}\omega_{T_z}\\ 

 & & + \left[ 2t\varepsilon \left(I_{P_z} +I_{T_x} \right) \right] \omega_{T_z}\omega_{P_y} +\left[ \frac{1}{c\varepsilon} \left(I_{P_z} +I_{T_x} \right)- c\varepsilon I_{T_y} \right] \omega_{P_x}\omega_{T_y} \\ 

 & & + \left[s\varepsilon \left(I_{P_z} +I_{T_x} \right)\right]\left[ c\eta\dot{\omega}_{B_x} + s\eta\dot{\omega}_{B_y}- \omega_{P_y}\omega_{B_z}\right]\\ 

 & & +\left[c\varepsilon \left(I_{P_x} +I_{P_y} \right)- s\varepsilon \, t\varepsilon \left(I_{P_z} +2I_{T_x} \right)-\frac{1}{c\varepsilon}I_{P_z} \right]\omega_{P_y}\omega_{P_x}\\ 

 & & + s\varepsilon k_{Pvf}\omega_{P_x} + c\varepsilon k_{Pvf}\omega_{B_z} - c\varepsilon k_{Pcf} \, \mathrm{sign}\left( \frac{1}{c\varepsilon} \omega_{T_z} - t\varepsilon \omega_{P_x} - \omega_{B_z}\right)\\ 

 & & - c\varepsilon T_{U_{Pz}} +c\varepsilon T_{az}
\end{array}
\label{crossap}
\end{equation}

Finalmente sustituyendo ~\ref{eq:w_Px} y ~\ref{eq:w_Py} en \ref{crossap} llegamos a la ecuaci\'{o}n de la din\'{a}mica de elevaci\'{o}n cruzada.


