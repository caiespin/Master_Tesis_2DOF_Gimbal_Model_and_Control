%%%%%%%%%%%%%%%%%%%%%%%%%%%%%%%%%%%%%%%%%%%%%%%%%%%%%%%%%%%%%%%%%%%%%%%%%%%%%%%%%%%%%%%%%%%%%%
% Plantilla para tesis por David Luna se encuentra bajo una Licencia
% Creative Commons Atribuci�n-NoComercial-CompartirIgual 3.0 Unported.
% http://creativecommons.org/licenses/by-nc-sa/3.0/
%%%%%%%%%%%%%%%%%%%%%%%%%%%%%%%%%%%%%%%%%%%%%%%%%%%%%%%%%%%%%%%%%%%%%%%%%%%%%%%%%%%%%%%%%%%%%%

\chapter{Introducci\'on}
\section{Motivaci\'on}
En las ultimas dos d\'{e}cadas se ha presenciado un gran incremento en la utilizaci\'{o}n de veh\'{i}culos a\'{e}reos no tripulados (VANTs) en aplicaciones de comunicaciones y defensa. Mientras que con sistemas VANT de gran tama\~{n}o se logran grandes prestaciones esto tambi\'{e}n incrementa enormemente el costo. Consecuentemente existe mucho inter�s en el desarrollo de plataformas peque�as y de bajo costo que puedan realizar tareas normalmente asignadas a sistemas VANT de mayor tama�o como en operaciones de reconocimiento, vigilancia e inteligencia.

El uso de aeronaves cambi\'{o} como las fuerzas militares analizan el espacio de batalla proporcionando conocimiento de la situaci\'{o}n mucho m\'{a}s all\'{a} de las perspectivas de sus fuerzas terrestres y navales, es por este motivo y gracias al avance de la tecnolog\'{i}a en el campo de las aeronaves no tripuladas (VANTs) que existe una gran proliferaci�n de sistemas de reconocimiento, vigilancia e inteligencia aerotransportados. 

Los sistemas de reconocimiento, vigilancia e inteligencia (ISR) por sus siglas en ingl�s son aquellos que recolectan, procesan y diseminan informaci\'{o}n e inteligencia para apoyar las acciones del comandante, jugando un papel primordial en el soporte a operaciones militares. Algunos ejemplos de sistemas ISR incluyen sistemas de vigilancia y reconocimiento que van desde sat\'{e}lites a aeronaves tripuladas como el Lockheed U-2\footnote{Es un avi\'{o}n de vigilancia a gran altitud, monomotor y monoplaza, usado por la Fuerza A\'{e}rea de los Estados Unidos (USAF) y previamente por la Agencia Central de Inteligencia (CIA). Realiza misiones de vigilancia todo tiempo a altitudes superiores a 21.000 m.} y no tripuladas como el Global Hawk\footnote{Es un veh\'{i}culo a\'{e}reo no tripulado (UAV) empleado por la Fuerza A\'{e}rea de los Estados Unidos como una aeronave de vigilancia a\'{e}rea.}, hasta otros equipos de base terrestre, a\'{e}rea, mar\'{i}tima o espacial as\'{i} como equipos humanos de inteligencia. Los datos de inteligencia prove\'{i}dos por sistemas ISR pueden tomar muchas formas, incluyendo im\'{a}genes \'{o}pticas, de radar, infrarrojas u otras se\~{n}ales electr\'{o}nicas.

Uno de los principales objetivos de una plataforma ISR es entregar la mayor calidad de informaci\'{o}n posible, la cual es en muchos casos transmisi�n de v\'{i}deo en tiempo real. Existen muchos factores a tomar en cuenta en la calidad de la imagen proporcionada por estos sistemas, siendo uno de los m\'{a}s importantes la capacidad de apuntar de forma precisa el eje \'{o}ptico de la c\'{a}mara, tambi\'{e}n llamado l\'{i}nea de vista (LOS por sus siglas en ingl\'{e}s) hacia un objetivo fijo o m\'{o}vil. Para resolver este problema usualmente se emplea un sistema de suspenci\'{o}n card\'{a}n o gimbal\footnote{Consiste en aros con ejes conc\'{e}ntricos cuyos ejes forman un \'{a}ngulo recto, lo cual permite mantener la orientaci\'{o}n de un eje de rotaci\'{o}n en el espacio aunque su soporte se mueva. En este trabajo, por motivos de simplicidad, denominaremos como \textit{"gimbal"} al sistema completo de motores, suspenci\'{o}n card\'{a}n, c\'{a}mara y electr\'{o}nica.} con el sensor montado en el eslab\'{o}n interno del mismo. La motivaci\'{o}n de este trabajo es el estudio a fondo de estos sistemas, desde la comprensi\'{o}n de su din\'{a}mica, el dise�o del control de estabilizaci\'{o}n, hasta los retos pr\'{a}cticos que representa su dise\~{n}o y construcci\'{o}n.
 

\section{Problema General}
Para lograr la estabilizaci\'{o}n de l\'{i}nea de vista se emplean usualmente sistemas gimbal de dos ejes Pan-Tilt, con sensores inerciales colocados en el eslab\'{o}n interno, que sirven como realimentaci\'{o}n al sistema de control que mantiene al sensor \'{o}ptico apuntando hacia el objetivo, si bien no es el \'{u}nico m\'{e}todo, la pr\'{a}ctica ha demostrado ser el m\'{a}s efectivo \cite{3}. El dise�o de sistemas m\'{o}viles de seguimiento de alta precisi�n din�mica es dif\'{i}cil de conseguir debido principalmente a las perturbaciones causadas por el movimiento del veh\'{i}culo, las limitaciones del hardware, las incertidumbres en los par\'{a}metros inerciales del sistema y la naturaleza altamente no lineal de estos sistemas, no obstante el dise�o de un control robusto puede absorber gran parte de estas perturbaciones, logrando mantener las variables de control dentro de l\'{i}mites aceptables.

\section{Objetivos del Proyecto}

El objetivo principal de esta tesis es primeramente desarrollar el modelo matem\'{a}tico de un sistema gimbal de dos grados de libertad con una configraci\'{o}n Pan-Tilt, posteriormente dise�ar un control que permita estabilizar de forma precisa la l\'{i}nea de vista de la c\'{a}mara montada en el eslab\'{o}n interno del sistema y finalmente se desea construir un prototipo en el cual se pueda implementar el control dise\~{n}ado y as\'{i} obtener datos pr\'{a}cticos del algoritmo desarrollado.

\section{Organizaci�n de la Tesis}

Este trabajo esta dividido en 7 cap\'{i}tulos, los cap\'{i}tulos 1 y 2 son introductorios y cubren los conceptos b\'{a}sicos y las aplicaciones de los sistemas gimbal de dos grados grados de libertad para la estabilizaci\'{o}n de l\'{i}nea de vista, los cap\'{i}tulos del 3 al 6 abarcan el desarrollo, pruebas y resultados, El cap\'{i}tulo 7 contiene las conclusiones y el trabajo futuro.

Cap\'{i}tulo 1, Introducci\'{o}n, este cap\'{i}tulo ha introducido la motivaci\'{o}n, la problem\'{a}tica general y los objetivos del proyecto de tesis.

Cap\'{i}tulo 2, Antecedentes, provee informaci\'{o}n y conceptos importantes sobre el \'{a}rea a tratar, detalles sobre los sistemas de estabilizaci\'{o}n inercial y la forma en la que se integran en los veh\'{i}culos A\'{e}reos Aut\'{o}nomos as\'{i} como el entorno de operaci\'{o}n al que estar\'{i}a sometido el sistema y una revisi\'{o}n bibliogr\'{a}fica sobre trabajos de investigaci\'{o}n relevantes en el \'{a}rea de estabilizaci\'{o}n de l\'{i}nea de vista.

Cap\'{i}tulo 3, Din\'{a}mica y Cinem\'{a}tica, se desarrolla el modelado del sistema por el m\'{e}todo de Newton-Euler. Posteriormente se analiza a detalle la din\'{a}mica del sistema presentado las propiedades del mismo y su comportamiento bajo ciertas consideraciones de simetr\'{i}a.

Cap\'{i}tulo 4, Control, provee informaci\'{o}n a detalle del dise�o del sistema de control y se desarrollan varias simulaciones en las que se muestra la efectividad del algoritmo dise\~{n}ado al compararlo con un compensador proporcional integral com\'{u}nmente utilizado para el control de estos sistemas.

Cap\'{i}tulo 5, Implementaci\'{o}n, describe el prototipo dise�ado tanto su mec\'{a}nica como su electr\'{o}nica y el proceso de su construcci\'{o}n, adema\'{a}s se presentan la programaci\'{o}n y arquitectura del hardware.

Cap\'{i}tulo 6, Resultados, Se describe la experimentaci\'{o}n desarrollada para la obtenci\'{o}n de los datos y se presentan los resultados obtenidos.

Cap\'{i}tulo 7, Conclusiones y Trabajo futuro, Resume los resultados claves del proyecto y provee los puntos de partida para un trabajo adicional. Este es el cap\'{i}tulo m\'{a}s importante del trabajo, pues provee los m\'{u}ltiples puntos a partir de los cuales podr\'{i}a mejorarse el trabajo presentado en esta tesis enfoc\'{a}ndose en los aprendizajes y cada unos de los puntos clave que afectan el desempe\~{n}o de la plataforma desarrollada.   